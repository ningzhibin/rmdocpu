\PassOptionsToPackage{unicode=true}{hyperref} % options for packages loaded elsewhere
\PassOptionsToPackage{hyphens}{url}
%
\documentclass[
]{article}
\usepackage{lmodern}
\usepackage{amssymb,amsmath}
\usepackage{ifxetex,ifluatex}
\ifnum 0\ifxetex 1\fi\ifluatex 1\fi=0 % if pdftex
  \usepackage[T1]{fontenc}
  \usepackage[utf8]{inputenc}
  \usepackage{textcomp} % provides euro and other symbols
\else % if luatex or xelatex
  \usepackage{unicode-math}
  \defaultfontfeatures{Scale=MatchLowercase}
  \defaultfontfeatures[\rmfamily]{Ligatures=TeX,Scale=1}
\fi
% use upquote if available, for straight quotes in verbatim environments
\IfFileExists{upquote.sty}{\usepackage{upquote}}{}
\IfFileExists{microtype.sty}{% use microtype if available
  \usepackage[]{microtype}
  \UseMicrotypeSet[protrusion]{basicmath} % disable protrusion for tt fonts
}{}
\makeatletter
\@ifundefined{KOMAClassName}{% if non-KOMA class
  \IfFileExists{parskip.sty}{%
    \usepackage{parskip}
  }{% else
    \setlength{\parindent}{0pt}
    \setlength{\parskip}{6pt plus 2pt minus 1pt}}
}{% if KOMA class
  \KOMAoptions{parskip=half}}
\makeatother
\usepackage{xcolor}
\IfFileExists{xurl.sty}{\usepackage{xurl}}{} % add URL line breaks if available
\IfFileExists{bookmark.sty}{\usepackage{bookmark}}{\usepackage{hyperref}}
\hypersetup{
  pdftitle={MetaLab MS identification Quick Summary},
  pdfauthor={Suggestions to imetalabca@gmail.com},
  pdfborder={0 0 0},
  breaklinks=true}
\urlstyle{same}  % don't use monospace font for urls
\usepackage[margin=1in]{geometry}
\usepackage{longtable,booktabs}
% Allow footnotes in longtable head/foot
\IfFileExists{footnotehyper.sty}{\usepackage{footnotehyper}}{\usepackage{footnote}}
\makesavenoteenv{longtable}
\usepackage{graphicx,grffile}
\makeatletter
\def\maxwidth{\ifdim\Gin@nat@width>\linewidth\linewidth\else\Gin@nat@width\fi}
\def\maxheight{\ifdim\Gin@nat@height>\textheight\textheight\else\Gin@nat@height\fi}
\makeatother
% Scale images if necessary, so that they will not overflow the page
% margins by default, and it is still possible to overwrite the defaults
% using explicit options in \includegraphics[width, height, ...]{}
\setkeys{Gin}{width=\maxwidth,height=\maxheight,keepaspectratio}
\setlength{\emergencystretch}{3em}  % prevent overfull lines
\providecommand{\tightlist}{%
  \setlength{\itemsep}{0pt}\setlength{\parskip}{0pt}}
\setcounter{secnumdepth}{5}
% Redefines (sub)paragraphs to behave more like sections
\ifx\paragraph\undefined\else
  \let\oldparagraph\paragraph
  \renewcommand{\paragraph}[1]{\oldparagraph{#1}\mbox{}}
\fi
\ifx\subparagraph\undefined\else
  \let\oldsubparagraph\subparagraph
  \renewcommand{\subparagraph}[1]{\oldsubparagraph{#1}\mbox{}}
\fi

% set default figure placement to htbp
\makeatletter
\def\fps@figure{htbp}
\makeatother


\title{MetaLab MS identification Quick Summary}
\author{Suggestions to
\href{mailto:imetalabca@gmail.com}{\nolinkurl{imetalabca@gmail.com}}}
\date{Report generated @2021-03-05 00:17:20}

\begin{document}
\maketitle

{
\setcounter{tocdepth}{4}
\tableofcontents
}
\hypertarget{intro}{%
\section{Intro}\label{intro}}

\textbf{This report provides some overall description of the database
search. } \textbf{Users can use this to quickly check the overal quality
of the experiment}

\hypertarget{take-home-figures}{%
\section{Take-home figures}\label{take-home-figures}}

\begin{itemize}
\tightlist
\item
  \textbf{Peptide Sequences Identified in total: }

  \begin{longtable}[]{@{}rl@{}}
  \toprule
  Pep & tide Sequences Identified\tabularnewline
  \midrule
  \endhead
  values & 31477\tabularnewline
  \bottomrule
  \end{longtable}
\end{itemize}

\begin{itemize}
\tightlist
\item
  \textbf{Avearge ms/ms identification rate(\%): }

  \begin{longtable}[]{@{}rl@{}}
  \toprule
  MS/ & MS Identified {[}\%{]}\tabularnewline
  \midrule
  \endhead
  values & 32.73\tabularnewline
  \bottomrule
  \end{longtable}
\end{itemize}

\begin{itemize}
\tightlist
\item
  \textbf{No meta information provided}
\end{itemize}

\hypertarget{msms-id-rate}{%
\section{MSMS id rate}\label{msms-id-rate}}

Why you should pay attention to MSMS Id rate?

\begin{enumerate}
\def\labelenumi{\arabic{enumi}.}
\item
  MS ID rate is a good repretation of the MS run quality. Raw files from
  Q-Exactive series should have roughly around 50\% ms ID rate
  (Percentage of MSMS spectra identified as peptided, at a 1\% FDR) for
  humane cell culture digest, and at least 20\% for metaproteomics
  samples according to experience.
\item
  MS ID rate should also be well-reproduced across samples and
  groupings.
\end{enumerate}

\begin{itemize}
\tightlist
\item
  Check the raw files if they have obnormally low ID rate, usually with
  abnormal LC/basepeak profile or low MS intensity.
\item
  A decreasing MS ID treand along sample running order indicates a
  performance drop of the MS: your MS might need to be cleaned. If the
  performance drops a lot, more than 20\% within running time for the
  whole project, without scramble of the sample run-order, the data
  might not be usable, unless very careflly caliburated.
\end{itemize}

\begin{center}\rule{0.5\linewidth}{0.5pt}\end{center}

\hypertarget{htmlwidget-ad316157be37b4a670ca}{}

\begin{center}\rule{0.5\linewidth}{0.5pt}\end{center}

\hypertarget{htmlwidget-6e37fb6c4f29746392e2}{}

\begin{center}\rule{0.5\linewidth}{0.5pt}\end{center}

\hypertarget{htmlwidget-795767f34d5ddfe65de6}{}

\hypertarget{peptide-sequence}{%
\section{Peptide Sequence}\label{peptide-sequence}}

\begin{center}\rule{0.5\linewidth}{0.5pt}\end{center}

\hypertarget{htmlwidget-f9c43329702a9093c11d}{}

\begin{center}\rule{0.5\linewidth}{0.5pt}\end{center}

\hypertarget{htmlwidget-f56c52042aa6e0d6bae2}{}

\begin{center}\rule{0.5\linewidth}{0.5pt}\end{center}

\hypertarget{htmlwidget-58f215df7abd6ef735dd}{}

\hypertarget{overall-performance}{%
\section{Overall Performance}\label{overall-performance}}

\textbf{check the overall performance for all raw files:}

\hypertarget{htmlwidget-7497ab2ac5bf31a5b35b}{}

\end{document}
